\documentclass[]{article}

% Imported Packages
%------------------------------------------------------------------------------
\usepackage{amssymb}
\usepackage{amstext}
\usepackage{amsthm}
\usepackage{amsmath}
\usepackage{enumerate}
\usepackage{fancyhdr}
\usepackage[margin=1in]{geometry}
\usepackage{graphicx}
\usepackage{extarrows}
\usepackage{setspace}
%------------------------------------------------------------------------------

% Header and Footer
%------------------------------------------------------------------------------
\pagestyle{plain}  
\renewcommand\headrulewidth{0.4pt}                                      
\renewcommand\footrulewidth{0.4pt}                                    
%------------------------------------------------------------------------------

% Title Details
%------------------------------------------------------------------------------
\title{Deliverable \#2 Group \#1 T02}
\author{SE 3A04: Software Design II -- Large System Design}
\date{}                               
%------------------------------------------------------------------------------

% Graphix
%------------------------------------------------------------------------------
\graphicspath{ {figures/} }

% Document
%------------------------------------------------------------------------------
\begin{document}

\maketitle	

\section{Introduction}
\label{sec:introduction}
% Begin Section

\subsection{Purpose}
\label{sub:purpose}
% Begin SubSection
The purpose of the document is to provide a high level design for the system architecture and class composition. This document is intended for use by the stakeholders of the project as reference throughout the various stages of the SDLC. The primary stakeholders include Prof. Ridha Khedri, who has commissioned this project as part of Software Engineering course SE3A04 as provided by McMaster University, Tutorial Assistants for SE3A04, who are responsible with oversight and review of the deliverables of the project and members of the development team (ie. authors of the document). Other stakeholders include any future teams of developers and managers, that may maintain the application or use this project application as an open source reference.

% End SubSection

\subsection{System Description}
\label{sub:system_description}
% Begin SubSection
Climatar is a world simulation software system illustrating the long term and short term effects of actions with respect to climate, greenhouse gas levels, economic stability and social relations with the governing bodies in the model world. The model world is based in the Avatar: The Last Airbender universe. Users are given governance of one of the four tribes in the Avatar universe: Air, Water, Fire, and Earth. Based on the state of the world and the regions that compose it, news events will be posed to the user that require action. Depending on their decisions the world will morph around the change and consequences will propagate throughout the simulation. All decisions have consequences associated with them, and the user can only react to events pertaining to their region, removing the omnipotent aspect of control, users must react to the changing world dynamically.
% End SubSection

\subsection{Overview}
\label{sub:overview}
% Begin SubSection
The remainder of the document outlines the high level system design specifically outlining how the system will be structured by outlining the Use Case Diagram, Analysis Class Diagram, the inherent Architecture Design, and Class Responsibilities. The document first outlines the use cases for Climatar. The use cases illustrate the business events pertaining to the system and layout the system functionality. Descriptions of each use case aide in formalizing the system functionality from the view points identified previously in the SRS. From the Use Cases a Noun-Verb Analysis was completed to create an Analysis Class Diagram. The Analysis Class Diagram acts as a starting point for the major classes the system will be composed of. The following section, Architecture Design, utilizes the information gained from the analysis class diagram to develop a system architecture to layout the system is a way such that each sub system has high cohesion and low coupling. Finally a Class Responsibility Collaboration (CRC) Card is developed for each class to formally outline the responsibilities, and dependencies for a given class.
% End SubSection

% End Section

\section{Use Case Diagram}
\label{sec:use_case_diagram}
% Begin Section

This section should provide a use case diagram for your application. 
\begin{enumerate}[a)]
	\item Each use case appearing in the diagram should be accompanied by a text description. 
\end{enumerate}
% End Section

\section{Analysis Class Diagram}
\label{sec:analysis_class_diagram}
% Begin Section
This section should provide an analysis class diagram for your application.
% End Section


\section{Architectural Design}
\label{sec:architectural_design}
% Begin Section
%This section should provide an overview of the overall architectural design of your application. You overall architecture should show the division of the system into subsystems with high cohesion and low coupling.

\subsection{System Architecture}
\label{sub:system_architecture}
% Begin SubSection
The system is divided into a GUI, World and sub system. The World subsystem further divides into Weather, Green House Gases (GHG), Political, and News Event subsystems. The GUI and World sub systems connect to one another but their controls run asynchronously to one another. The selected software architecture for the development of Climatar is Presentation-Abstraction-Control, PAC. 

PAC is optimal for the software architecture for Climatar for the aspects as it helps abstract the subsystems and allow for an ease of concurrency between the subsystems, as well as ensuring all modules are loosely coupled such that changes do not propagate through the system. This forces the system to be in a state of being easily maintainable post development. Furthermore, the project has a development time line of approximately one week for the coding portion. PAC allows developers to work in contained subsystems independent of other developers, thus the parallelism for development is optimal.

\begin{figure}[ht!]
\centering
\includegraphics[width=150mm]{ClimatarPACArchitecture}
\caption{System Architecture, PAC\label{pacarch}}
\end{figure}

% End SubSection

\subsection{Subsystems}
\label{sub:subsystems}
% Begin SubSection
The following subsystems compose Climatar:
\begin{itemize}
	\item \textbf{Application:} The Application subsystem is the entry point for the system, and acts as a top level controller for Climatar. This subsystem responds to user stimulus and passes command to the subsystem responsible for handling the stimulus, however this module can only command the World, Menu, and Play subsystems, which compose the control of the remaining subsystems.
	
	\item \textbf{Play:} The Play subsystem is responsible for displaying all elements composing the games view. Control is passed to this subsystem from Application.
	
	\item \textbf{Menu:} The Menu subsystem is responsible for all controls and UI components associated with the Main Menu/Title Screen for Climatar, and the game mode the game will be activated in. The Menu is given control by the Application subsystem and can control the FileHandler for loading and saving game instance requests.
	
	\item \textbf{FileHandler:} The FileHandler subsystem is controlled by the Menu for load and save game requests. 
	
	\item \textbf{World:} The World subsystem is responsible for simulating the world in a Climatar game instance, this includes the information retrieval and interpretation from all world creating subsystems. Regions/Nations are linked to their corresponding subsystems.
	
	\item \textbf{News:} The News subsystem is responsible for storing possible news events, and responding to news event requests through passing an appropriate event given a specified criteria.
	
	\item \textbf{Map:} The Map subsystem is responsible for the generation and holding of the current map for a given game instance.
	
	\item \textbf{Weather:} The Weather subsystem is responsible for the simulation of the weather of a region. This subsystem can be disconnected from the major system and not effect its ability to work, weather will just not be a factor in the simulations.
	
	\item \textbf{GHG:} The GHG subsystem is responsible for the simulation of the green house gas levels and the rate of change of green house gas levels of a region. This subsystem can be disconnected from the major system and not effect its ability to work, green house gases will just not be a factor in the simulations.
	
	\item \textbf{Political:} The Political subsystem is responsible for the simulation of the economical aspects of a region and a nations relations with the other nations. This subsystem can be disconnected from the major system and not effect its ability to work, political factors will simply not partake the simulations.
	
\end{itemize}
A visual representation of the high level system composition can be seen in Figure \ref{usesrelation}.
\begin{figure}[ht!]
\centering
\includegraphics[width=120mm]{ClimatarUsesRelation}
\caption{Module Uses Relation \label{usesrelation}}
\end{figure}
% End SubSection

% End Section
	
\section{Class Responsibility Collaboration (CRC) Cards}
\label{sec:class_responsibility_collaboration_crc_cards}
% Begin Section
This section should contain all of your CRC cards.

\begin{enumerate}[a)]
	\item Provide a CRC Card for each identified class
	\item Please use the format outlined in tutorial, i.e., 
	\begin{table}[ht]
	
		% NewsEventGenerator
		\centering
		\begin{tabular}{|p{10cm}|p{4cm}|}
		\hline 
		 \multicolumn{2}{|l|}{\textbf{Class Name: NewsEventGenerator}} \\
		\hline
		\textbf{Responsibility:} & \textbf{Collaborators:} \\
		\hline
		Knows how to generate a news event. & NewsEvent \\
		Can pass a news event to the NewsEventControl to be displayed. & NewsEventControl \\
		Can forward a news events selected action to the WorldSimulator. & WorldSimulator \\ 
		\hline
		\end{tabular}

		% NewsEvent
		\begin{tabular}{|p{10cm}|p{4cm}|}
		\hline 
		 \multicolumn{2}{|l|}{\textbf{Class Name: NewsEvent}} \\
		\hline
		\textbf{Responsibility:} & \textbf{Collaborators:} \\
		\hline
		Knows the textual description of the news event. & \\
		Knows a set of actions that can the user can select in response to the news event. & \\ 
		\hline
		\end{tabular}
		
		
	\end{table}
	
\end{enumerate}
% End Section

\appendix
\section{Division of Labour}
\label{sec:division_of_labour}
% Begin Section
Include a Division of Labour sheet which indicates the contributions of each team member. This sheet must be signed by all team members.

\begin{tabular}{ | l | l | }
\hline
	\textbf{Contributor Name} & \textbf{Contributions}  \\
  	\hline
  	Wenbin Yuan & \\		
  	\hline
  	Haris Khan &  \\
  	\hline
  	Riley McGee & Original draft of 1.* and 4.* including figures for Module Uses Relation and Architecture\\
  	\hline
  	Vishesh Gulatee & \\
  	\hline
  	James Taylor & \\
  	\hline
\end{tabular}
\\
\\
By signing below you agree to the work divisions stated above correctly representing all contributions made:


% End Section

\newpage
\section*{IMPORTANT NOTES}
\begin{itemize}
%	\item You do \underline{NOT} need to provide a text explanation of each diagram; the diagram should speak for itself
	\item Please document any non-standard notations that you may have used
	\begin{itemize}
		\item \emph{Rule of Thumb}: if you feel there is any doubt surrounding the meaning of your notations, document them
	\end{itemize}
	\item Some diagrams may be difficult to fit into one page
	\begin{itemize}
		\item It is OK if the text is small but please ensure that it is readable when printed
		\item If you need to break a diagram onto multiple pages, please adopt a system of doing so and thoroughly explain how it can be reconnected from one page to the next; if you are unsure about this, please ask about it
	\end{itemize}
	\item Please submit the latest version of Deliverable 1 with Deliverable 2
	\begin{itemize}
		\item It does not have to be a freshly printed version; the latest marked version is OK
	\end{itemize}
	\item If you do \underline{NOT} have a Division of Labour sheet, your deliverable will \underline{NOT} be marked
\end{itemize}


\end{document}
%------------------------------------------------------------------------------