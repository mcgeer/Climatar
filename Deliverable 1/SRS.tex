\documentclass[]{article}

% Imported Packages
%------------------------------------------------------------------------------
\usepackage{amssymb}
\usepackage{amstext}
\usepackage{amsthm}
\usepackage{amsmath}
\usepackage{enumerate}
\usepackage{fancyhdr}
\usepackage[margin=1in]{geometry}
\usepackage{graphicx}
\usepackage{extarrows}
\usepackage{setspace}
\usepackage{hyperref}
\usepackage{etoolbox}
\patchcmd{\thebibliography}{\section*{\refname}}{}{}{}
%------------------------------------------------------------------------------

% Header and Footer
%------------------------------------------------------------------------------
\pagestyle{plain}  
\renewcommand\headrulewidth{0.4pt}                                      
\renewcommand\footrulewidth{0.4pt}                                    
%------------------------------------------------------------------------------

% Title Details
%------------------------------------------------------------------------------
\title{Deliverable \#1 T02 Group 2}
\author{SE 3A04: Software Design II -- Large System Design}
\date{\today}                               
%------------------------------------------------------------------------------

% Graphics
%------------------------------------------------------------------------------
\graphicspath{ {figures/} }
%------------------------------------------------------------------------------

% Document
%------------------------------------------------------------------------------
\begin{document}

\maketitle	

\section{Introduction}
\label{sec:introduction}
% Begin Section
\subsection{Purpose}
\label{sub:purpose}
% Begin SubSection
	The purpose of the document is to provide a detailed description of the requirements of the application "Climatar". It serves to highlight objectives and key features, the application is perceived to embody, while also describing the scope, constraints and interface of the applicarion. This document is intended for the members of the development team (ie. authors of the document) as reference throughout the various stages of the SDLC.
% End SubSection

\subsection{Scope}
\label{sub:scope}
% Begin SubSection
	“Climatar” is a UI based mobile game, designed for Android OS, that simulates climate change as experienced by the world made popular by animated TV series, Avatar: The Last Airbender. The game generates news events, which describe real world actions, policies and decisions, which either contribute to or deter climate change, influencing variables that monitor the environmental sphere of the world.These news events are influenced by factors such as energy consumption, political stance, lifestyle and economic motivation and the enactment of these events are determined by the user.
 
	\cite{WoA}The geography of the model world is divided into four independent nations: Air Nomads, Water Tribe, Fire Nation and Earth Kingdom. The user has the option to chose one of these nations, enacting the news events pertaining to chosen nation or to play god, enacting the news events pertaining to all four nations. Each nation follows a unique political system, culture and lifestyle, and uses the element specified in its name as its primary source of energy. These elements are utilized by special portion of the population known as benders, who have the ability to naturally manipulate a specific element. The game ends once the model world has experienced sufficient damage due to climate change, severly changing the world's eco-system. 

	The application shall provide users with a save and load game option, allowing the user to play the game, as when they please. The game should be available for download through an application store or public repository.	
% End SubSection

\subsection{Definitions, Acronyms, and Abbreviations}
\label{sub:definitions_acronyms_and_abbreviations}
% Begin SubSection

% End SubSection

\subsection{References}
\label{sub:references}
% Begin SubSection
\bibliographystyle{ieeetran}
\bibliography{SRS}
% End SubSection

\subsection{Overview}
\label{sub:overview}
% Begin SubSection
	The rest of the document is divided into three sections. The second section provides an abstract overview of the product and its major functionality. The section defines the constraints for the system and the assumptions that affect the system’s requirements, all the while highlighting the proposed system's benefits that distinguish it from simillar products available. The third section provides a detailed description of the functional requirements perceived, which describe the specific behavior of the system. The contents of this section are classified as business events and viewpoints. The fourth and final section provides a detailed description of the non functional requirements, which provide a detailed description on specific areas of operation of the system.  

% End SubSection

% End Section

\section{Overall Description}
\label{sec:overall_description}
% Begin Section

This section outlines all details prefacing the requirements for the Climatar project. These details are to be understood prior to the functional and non-functional requirements for the system.

\subsection{Product Perspective}
\label{sub:product_perspective}
% Begin SubSection

Climate and world simulation as a software solution is not a new idea, however Climatar as a solution to world events is distinct from the current off-the-shelf solutions, appealing to a new client base, as well as utilizing current solution features to provide an enhanced software system for those using existing off-the-shelf solutions.

Existing products related to Climatar include: 
\begin{itemize}
	\item \textbf{BBC - Climate Challenge} 
		
		\cite{BBC}Is a web application for simulating the effects on controlling Great Britain. The application advances with 10 year turns, each turn allows a specific set of decisions to be made. Each decision effects just Great Britain. The overall goal is to become a no-emissions country within 200 years. \\
		Climatar contrasts this application through the following points:
		\begin{itemize}
			\item Climatar is for use on android devices. Android web players do not support the application BBC - Climate Challenge.
			\item Climatar allows different political regions to influence the world attributes, out of the control of the user.
			\item The over arching goal is to show feasible world reactions to events, elimination of C02 emissions thus is not the only simulation feature.
			\item Each play through of Climate Challenge is identical, at each phase of the simulation the same options are available. Climatar is to engage users interaction and knowledge growth by ensuring different instances of the simulation will play through differently based on previous actions, illustrating long term effects of decisions.
		\end{itemize}
	
	\item \textbf{Climate Models of Earth}

		Software illustrations of Climate predictions such as the Simple Climate Model by Monash, utilize mathematical predictions through data extrapolation on climate change to illustrate predictions on future climate if all growth rates remain constant with population, C02 levels, average temperature, etc\cite{ClimM}.\\
	Climatar contrasts this application through the following points:
		\begin{itemize}
			\item Climatar simulations do not assume constant growth rates, instead all system components like climate simulate and react with on another, causing complex predictions to events to be explored.
			\item Applications that model the earth are designed for  research use. Climatar is to educate the general public on the effects actions have both long and short term with respect to complex systems which could include, but is not limited to: environment, political toil, sea level, etc.
			\item The simulations executed in Climatar will however try and be as accurate to real physical systems, allowing for user analysis of reactions to events in a similar to real world way.
		\end{itemize}	 

	\item \textbf{Energy Wars} 
		Is a Android game\cite{EW} where users control the worlds demand for energy through development of green, and pollutant energy generation schemas. Event responses include natural disasters and war. The game itself is designed to be a puzzle to solve within each provided stage.\\
		Climatar contrasts this application through the following points:
		\begin{itemize}
			\item Climatar is to be a simulation solution of the complex systems interacting and handling aspects of regional events and how they effect other regions, whereas Energy wars is designed as a game to and the simulation of the world in the game is non realistic, such that the game moves at a fast pace.
			\item Both systems utilize the Android platform.
			\item The user of Climatar is not omnipotent, they may only react to the changing system through events, whereas in Energy Wars users control energy production of every single region, advancements in one relate to advancements in another.
			\item As opposed to the game modelling the earth, Climatar Models the world present in the Avatar universe. Changing the setting removes the restrictions to the real world, allowing for  simulations of generated regions to occur not as the world is, forcing adaptability to the game play aspects of Climatar.
		\end{itemize}
\end{itemize}

Although many more solutions exist, the above outline all of the main elements to current systems which contrast to Climatar's place in the family of world simulation software solutions.\\

Climatar is a self-contained software solution with respect to the Android system it is running on. No external databases, servers, or systems are required for Climatar to work. The System itself contains layers for world containment, world components (i.e. Climate, Greenhouse gas levels and emissions, political stances etc.), and event interaction. Depending on the simulated world, each of the sub layers interact differently. Furthermore the System interacts with the android platform utilizing the features available to applications on android devices. The following diagram illustrates the interaction of layers within the system:
\begin{figure}[h]
  \centering
  \includegraphics[width=14cm]{SRS_Interaction_Layers}
  \caption{High Level System Interaction Layers.}
\end{figure}
\newpage
% End SubSection

\subsection{Product Functions}
\label{sub:product_functions}
% Begin SubSection
The functionality Climatar is to present as a product is:
%List of Product functions for Climatar
\begin{enumerate}
	\item Climatar is to present the ability to simulate regions of a world based on regional, political, general climate, and pollution levels of the world under simulation.
	\item As a part of simulating the world, news items are presented as attributes of the world, it is up to the user to react to news.
	\item User interactions occur with an emphasis on not being omnipotent in how the world works, only in reactions to events within the world itself. In short users are given the ability to react to the changing world, propagating further changes, however they are not given the ability to command all aspects of the world.
	\item The application presents the user information allowing them to make needed decisions.
	\item Climatar allows for regional analysis, showing specific attributes of regions allowing for contrast in benefits versus disadvantages of regional reactions to changes made.
	\item Climatar is a tool to have users think about reactions to events on a global scale in a non real-time scale to show long term effects to event reactions unfold quickly.
\end{enumerate}
% End SubSection

\subsection{User Characteristics}
\label{sub:user_characteristics}
% Begin SubSection
The general characteristics of a user of Climatar are as follows:
\begin{itemize}
	\item Able to read and comprehend English.
	\item Understands how to navigate and use an Android system, as 	well as the controls provided by an Android system.
	\item The ability to see the system; including:
	\begin{enumerate}[-]
		\item Ability to view objects, and distinct elements on the device.
		\item Differentiate colouration of components such as red from blue.
	\end{enumerate}
\end{itemize}
% End SubSection

\subsection{Constraints}
\label{sub:constraints}
% Begin SubSection
\begin{enumerate}
	\item Climatar is to be restricted to android platforms.
	\item Programming language(s) used must be supported by SL4A
	\item  Climatar must implement a system that is composed of at least three sub-systems, each dealing with diverse environments from which the stimuli are received.
	\item Each sub-system must be modularized and easily swappable.
	\item Climatar must be able to customize the views to a specific set of sub-systems on the fly.
	\item  Climatar must react to specific stimuli in a way that respects the modelled domain.  Each subsystem must have at least one possible stimuli.
\end{enumerate}
% End SubSection

\subsection{Assumptions and Dependencies}
\label{sub:assumptions_and_dependencies}
% Begin SubSection
The following list of assumptions hold for the entirety of the SRS, any changes to the below list must be reflective throughout the SRS for it the be considered stable.
\begin{itemize}
	\item The use of the system assumes that the hardware utilizing the software is running at least Android KitKat 4.4.
	\item Use of the Java Development Kit 1.8.0 features are available to the system.
	\item Hardware under use is assumed to have a resolution of at least 720p.
	\item Climatar is to be granted any system accessibility requests by the user for the system to respond as stated to all requirements.
	\item Battery life of devices is assumed to remain consistent, the program assumes that the phone does not enter a power down state suddenly.
	\item Climatar is not planned for release on other platforms apart from Android KitKat 4.4+. Thus all requirements for the system do not include any need for porting the application beyond newer Android versions.
\end{itemize}
% End SubSection

\subsection{Apportioning of Requirements}
\label{sub:apportioning_of_requirements}
% Begin SubSection
Requirements being delayed, thus not included within the SRS are as follows:
\begin{enumerate}
	\item Online capabilities of the system are not included within the requirements however are alluded to as possible features.
\end{enumerate}
% End SubSection

% End Section


\section{Functional Requirements}
\label{sec:functional_requirements}
% Begin Section

\begin{enumerate}[{BE}1.]
	\item Start Application
	\begin{enumerate}[{VP1}.1]
		\item User Viewpoint
			\begin{enumerate}
				\item N/A
			\end{enumerate}
		\item Hardware Viewpoint
			\begin{enumerate}
				\item The underlying hardware shall create an graphics context. If the device is unable to create a graphics context, the application should close gracefully with an appropriate error message.
				\item The underlying hardware shall be initialized to respond to any touch input from the user.
				\item The hardware shall be able to load files that contain user data and save data, If the hardware fail to load the required files, the application should ask user to create a new game.
				\item The hardware should be able to render the start menu and refresh at realtime (24 FPS). 
			\end{enumerate}
		\item Game World Viewpoint
			\begin{enumerate}
				\item N/A
			\end{enumerate}
			
	\end{enumerate}
	\item Terminate Application
	\begin{enumerate}[{VP2}.1]
	\item User Viewpoint
			\begin{enumerate}
				\item The user should be capable of quitting the game at any time using the back button.

				\item The user should have the option of saving a game at any time.
				\item The user should be capable of enabling an autosaving option.
			\end{enumerate}
		\item Hardware Viewpoint
			\begin{enumerate}
				\item The hardware shall dispose of the graphics context.
				\item Touch events received from the hardware shall be ignored.
				\item If the autosaving option is enabled, the hardware will save the game files if capable, else display an appropriate error notification.
			\end{enumerate}
		\item Game World Viewpoint
			\begin{enumerate}
				\item N/A
			\end{enumerate}
		%\item \dots
	\end{enumerate}
		\item Pause Game
	\begin{enumerate}[{VP3}.1]
		\item User Viewpoint
			\begin{enumerate}
				\item The user should be capable of saving the game while it is paused.
			\end{enumerate}
		\item Hardware Viewpoint
			\begin{enumerate}
				\item The graphics context shall be updated to display the pause screen.
				\item If the back button event is captured by the hardware, the game shall resume.
			\end{enumerate}
		\item Game World Viewpoint
			\begin{enumerate}
				\item The system shall stop generating game events.
				\item The system shall stop the game timer.
			\end{enumerate}
			
					%\item \dots
	\end{enumerate}
		\item Resume Game
	\begin{enumerate}[{VP4}.1]
		\item User Viewpoint
			\begin{enumerate}
				\item N/A
			\end{enumerate}
		\item Hardware Viewpoint
			\begin{enumerate}
				\item N/A
			\end{enumerate}
		\item Game World Viewpoint
			\begin{enumerate}
				\item The System shall start generating events.

				\item The System shall resume the timer.
			\end{enumerate}
			
		\end{enumerate}
		\item Start Game
	\begin{enumerate}[{VP5}.1]
		\item User Viewpoint
			\begin{enumerate}
				\item The user should be able load a previous saved game.
				\item The user should be able to start a new game by selecting a game mode.
				\item There shall be two game modes:
				\begin{enumerate}
					\item Survival mode
					\item Overlord mode
				\end{enumerate}
				\item The user should be allowed to respond to any game events by selecting ?YES? or ?NO? in the UI.
			\end{enumerate}
		\item Hardware Viewpoint
			\begin{enumerate}
				\item The hardware shall load the game file which is selected by the user.
				\item The hardware shall load any graphics assets which are required for rendering the game world.
				\item While the assets are loading, a loading screen shall be presented to the user.
				\item After all assets are loaded, the game world and any UI shall be rendered in the graphics context.
				\item The hardware shall update the graphics context to display any new game events that occur.
			\end{enumerate}
		\item Game World Viewpoint
			\begin{enumerate}
				\item The game world shall initialize data when a new game is started.
				\item If the game world is loaded, the game world state shall reflect that of the loaded game file.
				\item The game world shall begin the game timer.
				\item The game world shall simulate and begin generating game events once it is started or loaded, based on the current game world state.
				\item Game events shall be yes or no questions which result in an increase or decrease of the chaos, temperature and or money.
			\end{enumerate}			
		%\item \dots
	\end{enumerate}
	\item Move map
		\begin{enumerate}[{VP6}.1]
			\item User Viewpoint
			\begin{enumerate}
				\item The user should be able to move the map with the finger.
			\end{enumerate}
		\item Hardware Viewpoint
			\begin{enumerate}
				\item The hardware should be capable to accept user?s finger input.
				\item The graphics context shall be updated to display the movement of map with realtime (24FPS).
			\end{enumerate}
		\item Game World Viewpoint
			\begin{enumerate}
				\item N/A
			\end{enumerate}			
		%\item \dots
	\end{enumerate}
		\item Input of decision
	\begin{enumerate}[{VP7}.1]
		\item User Viewpoint
			\begin{enumerate}
				\item The user shall respond to game events by selecting ?YES? or ?NO?.
			\end{enumerate}
		\item Hardware Viewpoint
			\begin{enumerate}
				\item The hardware shall update the chaos, politics and money values on screen.
			\end{enumerate}
		\item Game World Viewpoint
			\begin{enumerate}
				\item The game world shall appropriately increase or decreases the chaos, politics, and money values based on the decision of the user.
				\item  The game world shall check for lose conditions by checking if the following is truth: (chaos > 100 percent OR money <= 0 OR temp >= 100 percent).
				\item If the lose conditions are truthy, the game shall display the gameover screen.

			\end{enumerate}
					%\item \dots
	\end{enumerate}
		\item Reset Game
	\begin{enumerate}[{VP9}.1]
		\item User Viewpoint
			\begin{enumerate}
				\item The user shall choose yes or now to confirm reseting the game world.
			\end{enumerate}
		\item Hardware Viewpoint
			\begin{enumerate}
				\item The hardware shall render and update the display with new map and initialized UI.
			\end{enumerate}
		\item Game World Viewpoint
			\begin{enumerate}
				\item The System shall initialize all data and generate new map.
				\item The System shall start generating game events.
			\end{enumerate}
			
		%\item \dots
	\end{enumerate}
		\item The game ends
	\begin{enumerate}[{VP9}.1]
		\item User Viewpoint
			\begin{enumerate}
				\item N/A
			\end{enumerate}
		\item Hardware Viewpoint
			\begin{enumerate}
				\item The graphics context shall be updated to display the game over screen with score, highest score, restart button and quit button.
				\end{enumerate}
		\item Game World Viewpoint
			\begin{enumerate}
				\item The system shall be terminated and all temporary data shall be cleaned.
			\end{enumerate}			
		%\item \dots
	\end{enumerate}
	
\end{enumerate}


% End Section

\section{Non-Functional Requirements}
\label{sec:non-functional_requirements}
% Begin Section

% manually managed enumi variable to preserve enumerations across sections
\newcounter{_enumi}
\newcommand{\holdEnum}{\setcounter{_enumi}{\value{enumi}}}
\newcommand{\resumeEnum}{\setcounter{enumi}{\value{_enumi}}}

\subsection{Look and Feel Requirements}
\label{sub:look_and_feel_requirements}
% Begin SubSection

\subsubsection{Appearance Requirements}
\label{ssub:appearance_requirements}
% Begin SubSubSection
\begin{enumerate}[{LF}1. ]
	\item The product's interface shall be designed in a way such that it can be easily adapted to different screen sizes.
	\item The product shall minimize the use of visual adornments that would otherwise clutter the interface.
	\item The product shall use common navigational elements (ie. status bar, tab bar, hamburger menu) to help usability amongst new users.
	\item The product shall use common iconography to help usability amongst new users. 
	\item The product shall make heavy use of contrast to help it's usability amongst colour blind users.
	\item The product shall predominantly use a single colour to help develop and maintain it's brand identity.
	\holdEnum
\end{enumerate}
% End SubSubSection

\subsubsection{Style Requirements}
\label{ssub:style_requirements}
% Begin SubSubSection
\begin{enumerate}[{LF}1.]
	\resumeEnum
	%TODO: Reference the google material design guidelines in section 1.4
	\item Where it can, the product shall conform to Google's material design guidelines.
	\item The product shall use primarily sans-serif fonts.
	\item The product's app icon shall be designed in the common negative-space style to help it sit better amongst other popular app icons.  
\end{enumerate}
% End SubSubSection

% End SubSection

\subsection{Usability and Humanity Requirements}
\label{sub:usability_and_humanity_requirements}
% Begin SubSection

\subsubsection{Ease of Use Requirements}
\label{ssub:ease_of_use_requirements}
% Begin SubSubSection
\begin{enumerate}[{UH}1. ]
	\item The product shall be usable by anyone fluent in the English language.
	\item The product's navigational elements should be found in under 2 seconds by new users.
	\item The product's navigational elements shall behave in the ways standard to the Android operating system. This should make the app more intuitive to use for new users. 
	\item The product's functions shall be designed a way such that they require a minimum number of gestures from the user.
	\item Where it makes sense, the product shall favour input elements such as sliders over raw text input.
	\holdEnum
\end{enumerate}
% End SubSubSection

\subsubsection{Personalization and Internationalization Requirements}
\label{ssub:personalization_and_internationalization_requirements}
% Begin SubSubSection
\begin{enumerate}[{UH}1. ]
	\resumeEnum
	\item The product shall favour the use of icons such that, when needed, it can be more easily adapted to different languages and cultures.
	\item The product shall use Android SDK best practices so that the user's global accessibility settings (text size, text-to-speech etc...) will take effect in the app. 
	\holdEnum
\end{enumerate}
% End SubSubSection

\subsubsection{Learning Requirements}
\label{ssub:learning_requirements}
% Begin SubSubSection
\begin{enumerate}[{UH}1. ]
	\resumeEnum
	\item By using standard Android navigational elements, the product should not take more than 10 minutes for a new user to become adept with it.
	\item For any non-traditional feature, the product shall use short tutorial text to educate the user about it's functionality.
	\holdEnum
\end{enumerate}
% End SubSubSection

\subsubsection{Understandability and Politeness Requirements}
\label{ssub:understandability_and_politeness_requirements}

% Begin SubSubSection
\begin{enumerate}[{UH}1. ]
 	\resumeEnum
	\item The product shall contain no language or ideas inappropriate for an elementary school level user.
	\holdEnum
\end{enumerate}
% End SubSubSection

\subsubsection{Accessibility Requirements}
\label{ssub:accessibility_requirements}
% Begin SubSubSe ction
\begin{enumerate}[{UH}1. ]
	\resumeEnum
	\item The product's interface shall use intractable elements at a minimum of 60x60points in size to help users with reduced motor skills.
	\item By employing standard Android SDK best practices as mentioned in LF7, the product shall adapt the user's global accessibility settings such as TTS (text to speech) and increased/decreased font sizes. 
	\item The product shall make heavy use of contrast and minimize the use of multiple colours to help colour blind users (This is also mentioned in LF5).
\end{enumerate}
% End SubSubSection

% End SubSection

\subsection{Performance Requirements}
\label{sub:performance_requirements}
% Begin SubSection

\subsubsection{Speed and Latency Requirements}
\label{ssub:speed_and_latency_requirements}
% Begin SubSubSection
\begin{enumerate}[{PR}1. ]
	\item Any network requests made by the product will be dispatched and listened to on a background thread so that the user interface (driven by the main thread) does not stall.
	\item Each user interaction shall be followed up by a visual change on screen. The time between these two events should never exceed half a second.
	\item Procedural world generation for a new game should take no longer than 2 seconds.
	\item Loading an in-progress game should take no longer than 1 second.
	\item The app should launch in no longer than 2 seconds.
	\item The app should resume from background in no longer than 1 second.
	\holdEnum
\end{enumerate}
% End SubSubSection

\subsubsection{Safety-Critical Requirements}
\label{ssub:safety_critical_requirements}
% pretty sure this is n.a - James
n.a
% Begin SubSubSection
%\begin{enumerate}[{PR}1. ]
%	\item 
%\end{enumerate}
% End SubSubSection

\subsubsection{Precision or Accuracy Requirements}
\label{ssub:precision_or_accuracy_requirements}
% Begin SubSubSection
\begin{enumerate}[{PR}1. ]
	\resumeEnum
	\item When performing floating-point arithmetic, the app shall favour the use double precision to minimize rounding error.
	\holdEnum
\end{enumerate}
% End SubSubSection

\subsubsection{Reliability and Availability Requirements}
\label{ssub:reliability_and_availability_requirements}
% Begin SubSubSection
\begin{enumerate}[{PR}1. ]
	\resumeEnum
	\item The product shall not require an active internet connection (ie. it will be available offline).
	\item In any case that the device is running under normal circumstances, the app shall be available.
	\item Under any normal use case, the app should not crash.
	\holdEnum
\end{enumerate}
% End SubSubSection

\subsubsection{Robustness or Fault-Tolerance Requirements}
\label{ssub:robustness_or_fault_tolerance_requirements}
% Begin SubSubSection
\begin{enumerate}[{PR}1. ]
	\resumeEnum
	\item The app shall temporarily save an ongoing game in the event that it is terminated unexpectedly by the user or by the operating system in the event that memory needs to be freed. 
	\item The app shall reopen the saved game from PR11 in the event that it is launched following an unexpected termination.
	\holdEnum
\end{enumerate}
% End SubSubSection

\subsubsection{Capacity Requirements}
\label{ssub:capacity_requirements}
% Begin SubSubSection
\begin{enumerate}[{PR}1. ]
	\resumeEnum
	\item The product shall allow the user to save up to 10 in-progress games. This does not include the game temporarily saved in the event that the app terminates unexpectedly (PR11).
	\holdEnum
\end{enumerate}
% End SubSubSection

\subsubsection{Scalability or Extensibility Requirements}
\label{ssub:scalability_or_extensibility_requirements}
% Begin SubSubSection
\begin{enumerate}[{PR}1. ]
	\resumeEnum
	\item The product is planned to feature an online versus mode in the future in which it will allow for up to 4 concurrent players per session. 
	\holdEnum
\end{enumerate}
% End SubSubSection

\subsubsection{Longevity Requirements}
\label{ssub:longevity_requirements}
% Begin SubSubSection
\begin{enumerate}[{PR}1. ]
	\resumeEnum
	\item The product should be maintainable and extensible within it's planned maintenance budget for 5 years post launch.
\end{enumerate}
% End SubSubSection

% End SubSection

\subsection{Operational and Environmental Requirements}
\label{sub:operational_and_environmental_requirements}
% Begin SubSection

\subsubsection{Expected Physical Environment}
\label{ssub:expected_physical_environment}
% Begin SubSubSection
\begin{enumerate}[{OE}1. ]
	\item As the app operates within the Android environment and introduces no restriction under which it operates within that environment, the product shall be usable under circumstances where the Android operating system allows it.
	\holdEnum
\end{enumerate}
% End SubSubSection

\subsubsection{Requirements for Interfacing with Adjacent Systems}
\label{ssub:requirements_for_interfacing_with_adjacent_systems}
% Begin SubSubSection
\begin{enumerate}[{OE}1. ]
	\resumeEnum
	\item The product shall work on any smartphone running Android 4.4 (KitKat) and later.
	\item As an online feature is planned for sometime in the future, the app shall take into consideration architectures for interfacing with a socket based web server.
	\item Future updates to the app shall not invalidate game saves made on previous versions.
	\holdEnum
\end{enumerate}
% End SubSubSection

\subsubsection{Productization Requirements}
\label{ssub:productization_requirements}
% Begin SubSubSection
\begin{enumerate}[{OE}1. ]
	\resumeEnum
	% TODO: Link to the google play store in section 1.4.
	\item The product shall be installable directly from the Google Play Store.
	\item The product shall have 5 screenshots and a short text description to advertise it on the Google Play Store.
	\item The product shall be no bigger than 15MB. 
	\holdEnum
\end{enumerate}
% End SubSubSection

\subsubsection{Release Requirements}
\label{ssub:release_requirements}
% Begin SubSubSection
\begin{enumerate}[{OE}1. ]
	\resumeEnum
	\item Maintenance and new content updates will be published at the end of each month.
	\holdEnum
\end{enumerate}
% End SubSubSection

% End SubSection

\subsection{Maintainability and Support Requirements}
\label{sub:maintainability_and_support_requirements}
% Begin SubSection

\subsubsection{Maintenance Requirements}
\label{ssub:maintenance_requirements}
% Begin SubSubSection
\begin{enumerate}[{MS}1. ]
	\item Hotfixes must available to publish within a day of them being made.
	\holdEnum
\end{enumerate}
% End SubSubSection

\subsubsection{Supportability Requirements}
\label{ssub:supportability_requirements}
% Begin SubSubSection
\begin{enumerate}[{MS}1. ]
	\resumeEnum
	% TODO: Link to the google play store in 1.4.
	\item A link to the company's support url will be provided in the description on the Google Play Store for feature requests and bug reports.
	\holdEnum
\end{enumerate}
% End SubSubSection

\subsubsection{Adaptability Requirements}
\label{ssub:adaptability_requirements}
% Begin SubSubSection
\begin{enumerate}[{MS}1. ]
	\resumeEnum
	\item The product is expected to run on version of Android 4.4 (KitKat) and later.
	\item The product may be ported to iOS and Windows Phone in the future.
	\item The product may be adapted to accommodate different screen sizes and aspect ratios in the future.
\end{enumerate}
% End SubSubSection

% End SubSection

\subsection{Security Requirements}
\label{sub:security_requirements}
% Begin SubSection

\subsubsection{Access Requirements}
\label{ssub:access_requirements}
% Begin SubSubSection
\begin{enumerate}[{SR}1. ]
	\item When online features are introduced in the future, only a user logged into an account may play under that account's identity.
	\holdEnum
\end{enumerate}
% End SubSubSection

\subsubsection{Integrity Requirements}
\label{ssub:integrity_requirements}
% Begin SubSubSection
\begin{enumerate}[{SR}1. ]
	\resumeEnum
	\item The product shall guard against the introduction of incorrect data to, for example, guard against an attacker spoofing high scores numbers.
	\item When online features are introduced in the future, the product shall guard against the automated generation of user accounts. 
	\holdEnum
\end{enumerate}
% End SubSubSection

\subsubsection{Privacy Requirements}
\label{ssub:privacy_requirements}
% Begin SubSubSection
\begin{enumerate}[{SR}1. ]
	\resumeEnum
	\item The product shall inform users of its information policy before collecting data on them.
	\item The product shall inform the users of changes to its information policy.
	\item The product shall adhere to it's information policy if revealing private user information.
	\holdEnum
\end{enumerate}
% End SubSubSection

\subsubsection{Audit Requirements}
\label{ssub:audit_requirements}
n.a
% Begin SubSubSection
%\begin{enumerate}[{SR}1. ]
%	\resumeEnum
%	\item 
%	\holdEnum
%\end{enumerate}
% End SubSubSection

\subsubsection{Immunity Requirements}
\label{ssub:immunity_requirements}

% Begin SubSubSection
\begin{enumerate}[{SR}1. ]
	\resumeEnum
	\item As the product runs under management of the Android operating system, the immunity of the product relies on the guards put in place by Android.
\end{enumerate}
% End SubSubSection

% End SubSection

\subsection{Cultural and Political Requirements}
\label{sub:cultural_and_political_requirements}
% Begin SubSection

\subsubsection{Cultural Requirements}
\label{ssub:cultural_requirements}
% Begin SubSubSection
\begin{enumerate}[{CP}1. ]
	\item The product shall aim to not offend any ethnic or religious groups.
	\item The product shall hold seasonal events celebrating holidays from a variety of ethnic and religious backgrounds.
	\holdEnum
\end{enumerate}
% End SubSubSection

\subsubsection{Political Requirements}
\label{ssub:political_requirements}
n.a
% Begin SubSubSection
%\begin{enumerate}[{CP}1. ]
%	\item 
%\end{enumerate}
% End SubSubSection

% End SubSection

\subsection{Legal Requirements}
\label{sub:legal_requirements}
% Begin SubSection

\subsubsection{Compliance Requirements}
\label{ssub:compliance_requirements}
% Begin SubSubSection
\begin{enumerate}[{LR}1. ]
	\item Personal information shall be implemented in compliance with the data protection act.
	\holdEnum
\end{enumerate}
% End SubSubSection

\subsubsection{Standards Requirements}
\label{ssub:standards_requirements}
% Begin SubSubSection
\begin{enumerate}[{LR}1. ]
	\resumeEnum
	\item The app will conform with acceptable performance standards for games and simulations.
	\item The app will conform to the Android Core App Quality standards to minimize turnover approval time and to increase the chances of being featured in the Google Play Store.
\end{enumerate}
% End SubSubSection

% End SubSection

% End Section

\appendix
\section{Division of Labour}
\label{sec:division_of_labour}
% Begin Section
Include a Division of Labour sheet which indicates the contributions of each team member. This sheet must be signed by all team members.
% End Section
\nocite{MDes}
\nocite{CAQ}
\newpage
\section*{IMPORTANT NOTES}
\begin{itemize}
	\item Be sure to include all sections of the template in your document regardless whether you have something to write for each or not
	\begin{itemize}
		\item If you do not have anything to write in a section, indicate this by the \emph{N/A}, \emph{void}, \emph{none}, etc.
	\end{itemize}
	\item Uniquely number each of your requirements for easy identification and cross-referencing
	\item Highlight terms that are defined in Section~1.3 (\textbf{Definitions, Acronyms, and Abbreviations}) with \textbf{bold}, \emph{italic} or \underline{underline}
	\item For Deliverable 1, please highlight, in some fashion, all (you may have more than one) creative and innovative features. Your creative and innovative features will generally be described in Section~2.2 (\textbf{Product Functions}), but it will depend on the type of creative or innovative features you are including.
\end{itemize}


\end{document}
%------------------------------------------------------------------------------