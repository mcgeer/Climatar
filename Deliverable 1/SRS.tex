\documentclass[]{article}

% Imported Packages
%------------------------------------------------------------------------------
\usepackage{amssymb}
\usepackage{amstext}
\usepackage{amsthm}
\usepackage{amsmath}
\usepackage{enumerate}
\usepackage{fancyhdr}
\usepackage[margin=1in]{geometry}
\usepackage{graphicx}
\usepackage{extarrows}
\usepackage{setspace}
\usepackage{hyperref}
\usepackage{etoolbox}
\patchcmd{\thebibliography}{\section*{\refname}}{}{}{}
%------------------------------------------------------------------------------

% Header and Footer
%------------------------------------------------------------------------------
\pagestyle{plain}  
\renewcommand\headrulewidth{0.4pt}                                      
\renewcommand\footrulewidth{0.4pt}                                    
%------------------------------------------------------------------------------

% Title Details
%------------------------------------------------------------------------------
\title{Deliverable \#1 T02 Group 2}
\author{SE 3A04: Software Design II -- Large System Design}
\date{\today}                               
%------------------------------------------------------------------------------

% Graphics
%------------------------------------------------------------------------------
\graphicspath{ {figures/} }
%------------------------------------------------------------------------------

% Document
%------------------------------------------------------------------------------
\begin{document}

\maketitle	

\section{Introduction}
\label{sec:introduction}
% Begin Section
\subsection{Purpose}
\label{sub:purpose}
% Begin SubSection
	The purpose of the document is to provide a detailed description of the requirements for the software application described in the scope. It serves to highlight objectives and key features of the application, while also describing constraints and the interface of the system-to-be, and it's environment. This document is intended for use by the stakeholders of the project as reference throughout the various stages of the SDLC. The primary stakeholders include Prof. Ridha Khedri, who has commissioned this project as part of Software Engineering course SE3A04 as provided by McMaster University, Tutorial Assistants for SE3A04, who are responsible with oversight and review of the deliverables of the project and members of the development team (ie. authors of the document). Other stakeholders include any future teams of developers and managers, that may maintain the application or use this project application as an open source reference. 
% End SubSection

\subsection{Scope}
\label{sub:scope}
% Begin SubSection
	“Climatar” is a UI based mobile game, designed for Android OS, that simulates climate change as experienced by the world made popular by animated TV series, Avatar: The Last Airbender. The game generates news events, which describe real world actions, policies and decisions, which either contribute to or deter climate change, influencing variables that monitor the environmental sphere of the applications model world. The objective of the game is to regulate news events, in order to minimize the effects of greenhouse gasses and global warming.

The world of Avatar has been selected as the application's model world as it provides a fictional backdrop for the game and the world's political, social, and economic environment describes a simpler model of the contemporary world. This ensures that the consequences of the news events are enforced on the world independent of prior news events. \cite{WoA}The geography of the model world is divided into four independent nations: Air Nomads, Water Tribe, Fire Nation and Earth Kingdom. The user will have the option to choose one of these nations, enacting the news events pertaining to the chosen nation through a role of governance. Actions taken on news events further enact long and short term consequences pertaining to all four nations. Each nation follows a unique political system, culture, and lifestyle; whilst using the element specified in its name as the nations primary source of energy. 

The objective of the game is to regulate news events in order to minimize the effects of greenhouse gasses and mitigate global warming. The game ends once the model world has experienced sufficient damage due to climate change, severely changing the world's ecosystem, and resulting the user with a loss. The application shall provide users with a save and load game option, allowing the user to play the game when they please without loss of data. The game should be available for download through an application store or public repository.	
% End SubSection

\subsection{Definitions, Acronyms, and Abbreviations}
\label{sub:definitions_acronyms_and_abbreviations}
% Begin SubSection
\begin{enumerate}
\item \textbf{BBC}: British Broadcasting Corporation, a public service broadcaster based in the UK.
\item \textbf{Climate Change}: Extended and permanent change, as experienced by long term weather patterns.
\item \textbf{GHG}: Greenhouse Gases, classification for gases in the atmosphere that absorb and emit radiation within the thermal infra-red light range. Major contributor of Global Warming.
\item \textbf{Global Warming}: Gradual increase in average surface temperature of Earth.
\item\textbf{FPS}: Frames per second,  frequency at which a series of images (frames) are displayed, consecutively.
\item \textbf{SDLC}: Software Development Life Cycle, described as the framework used to define the hierarchy of tasks to be performed at each stage of the software development process.
\item \textbf{SL4A}:  Scripting Layer for Android, library used for creating and executing various scripting languages based in Android systems.
\item \textbf{TTS}: Text to speech.
\end{enumerate}
% End SubSection

\subsection{References}
\label{sub:references}
% Begin SubSection
\bibliographystyle{ieeetran}
\bibliography{SRS}
% End SubSection

\subsection{Overview}
\label{sub:overview}
% Begin SubSection
	The rest of the document is divided into three main sections: Overall Description, Functional Requirements, and Non-Functional Requirements. The Overall description of the project provides an abstract overview of the product and its major functionality. The section defines the constraints for the system and the assumptions that affect the system's requirements, all the while highlighting the proposed system's benefits that distinguish it from similar products available. The Functional Requirements section provides a detailed description of the functional requirements perceived, which describe the specific behaviour of the system. The contents of this section are classified as business events and viewpoints. The final section, Non-functional Requirements, provides a detailed description of the non-functional requirements, which describe the quality attributes of the system. The plan for implementing functional and non-functional requirements will be detailed in system design and system architecture documents. 
% End SubSection

% End Section

\section{Overall Description}
\label{sec:overall_description}
% Begin Section
\subsection{Product Perspective}
\label{sub:product_perspective}
% Begin SubSection

Climate and world simulation as a software solution is not a new idea, however Climatar as a solution to world events is distinct from the current off-the-shelf solutions, appealing to a new client base, as well as utilizing current solution features to provide an enhanced software system for those using existing off-the-shelf solutions.

Existing products related to Climatar include: 
\begin{itemize}
	\item \textbf{BBC - Climate Challenge} 	
		\cite{BBC}Is a web application for simulating the effects on controlling taxation, laws, and government spending for the country of Great Britain. The major areas where the user may act are: military, food and water supplies, energy, the environment, and political relations. The application advances with 10 year turns, each turn allows a specific set of decisions to be made within the major areas the user governs. Each decision affects just Great Britain. The overall goal is to become a no-emissions country within 200 years. \\
		Climatar contrasts this application through the following points:
		\begin{itemize}
			\item Climatar is for use on android devices. Android web players do not support the application BBC - Climate Challenge.
			\item Climatar allows different political regions to influence the world attributes, out of the control of the user.
			\item The over arching goal is to show feasible world reactions to events, elimination of C02 emissions thus is not the only simulation feature.
			\item Each play through of Climate Challenge is identical, at each phase of the simulation the same options are available. Climatar is to engage users interaction and knowledge growth by ensuring different instances of the simulation will play through differently based on previous actions, illustrating long term effects of decisions.
		\end{itemize}
	
	\item \textbf{Climate Models of Earth}

		Software illustrations of Climate predictions such as the Simple Climate Model by Monash, utilize mathematical predictions through data extrapolation on climate change to illustrate predictions on future climate if all growth rates remain constant with population, C02 levels, average temperature, etc\cite{ClimM}.\\
	Climatar contrasts this application through the following points:
		\begin{itemize}
			\item Climatar simulations do not assume constant growth rates, instead all system components like climate simulate and react with on another, causing complex predictions to events to be explored.
			\item Applications that model the earth are designed for  research use. Climatar is to educate the general public on the effects actions have both long and short term with respect to complex systems which could include, but is not limited to: environment, political toil, sea level, etc.
			\item The simulations executed in Climatar will however try to be as accurate to real physical systems, allowing for user analysis of reactions to events in a similar to real world way.
		\end{itemize}	 

	\item \textbf{Energy Wars} 
		Is an Android game\cite{EW} where users control the worlds demand for energy through development of green energy schemas such as Wind, and Solar generation stations, as well as pollutant energy generation schemas such as Coal, Natural Gas, and Oil. The world is ever responding to the state of the worlds climate, such event responses include natural disasters and war. The game itself is designed to be a puzzle to solve within each provided stage. Providing points for quickness of completeness and the state of the world after a given challenge has been met.\\
		Climatar contrasts this application through the following points:
		\begin{itemize}
			\item Climatar is to be a simulation solution of the complex systems interacting and handling aspects of regional events and how they effect other regions, whereas Energy wars is designed as a game to and the simulation of the world in the game is non realistic, such that the game moves at a fast pace.
			\item Both systems utilize the Android platform.
			\item The user of Climatar is not omnipotent, they may only react to the changing system through events, whereas in Energy Wars users control energy production of every single region, advancements in one relate to advancements in another.
			\item As opposed to the game modelling the earth, Climatar Models the world present in the Avatar universe. Changing the setting removes the restrictions to the real world, allowing for  simulations of generated regions to occur not as the world is, forcing adaptability to the game play aspects of Climatar.
		\end{itemize}
\end{itemize}

Although many more solutions exist, the above outline all of the main elements to current systems which contrast to Climatar's place in the family of world simulation software solutions.\\

Climatar is a self-contained software solution with respect to the Android system it is running on. No external databases, servers, or systems are required for Climatar to work. The system itself contains layers for world containment, world components (i.e. Climate, Greenhouse gas levels and emissions, political stances etc.), and event interaction. Depending on the simulated world, each of the sub layers interact differently. Furthermore the System interacts with the Android platform utilizing the features available to applications on Android devices.
% End SubSection

\subsection{Product Functions}
\label{sub:product_functions}
% Begin SubSection
As an End User, Climatar is to present the ability to simulate regions of a world based on regional, political, general climate, and pollution levels of the world under simulation.  Users are granted the ability to make the decisions a country may be faced with based on the state of the world, and relations with other countries. However, the User is not granted a role of omnipotence, they cannot control all aspects of how the world or their region behaves. Instead, based on previous actions and the state of the country under governance, the User is presented with events, in the form of news items, which they can react to. Each event has some factor of consequence associated with it (such as: additional pollution, cost, loss of relation...), providing Climatar Users the ability to analyse the long term effects of decisions on a global and regional level. Lastly, information is presented to the user on a per region and international scale, allowing for decisions to be made with a maximized set of information to contrast.
% End SubSection

\subsection{User Characteristics}
\label{sub:user_characteristics}
% Begin SubSection
The general characteristics of a user of Climatar are as follows:
\begin{itemize}
	\item Able to read and comprehend English.
	\item Understands how to navigate and use an Android system, as 	well as the controls provided by an Android system.
	\item The ability to see the system; including:
	\begin{enumerate}[-]
		\item Ability to view objects, and distinct elements on the device.
		\item Differentiate colouration of components such as red from blue.
	\end{enumerate}
\end{itemize}
% End SubSection

\subsection{Constraints}
\label{sub:constraints}
% Begin SubSection
\begin{enumerate}
	\item Climatar is to be restricted to android platforms.
	\item Programming language(s) used must be supported by SL4A
\end{enumerate}
% End SubSection

\subsection{Assumptions and Dependencies}
\label{sub:assumptions_and_dependencies}
% Begin SubSection
The following list of assumptions hold for the entirety of the SRS, any changes to the below list must be reflective throughout the SRS for it the be considered stable.
\begin{itemize}
	\item The use of the system assumes that the hardware utilizing the software is running at least Android KitKat 4.4.
	\item Hardware under use is assumed to have a resolution of at least 720p.
	\item Climatar is to be granted any system accessibility requests by the user for the system to respond as stated to all requirements.
	\item The program assumes the power of the system is not immediately set to off without the operating system notifying all applications.
\end{itemize}
% End SubSection

\subsection{Apportioning of Requirements}
\label{sub:apportioning_of_requirements}
% Begin SubSection
Requirements being delayed, thus not included within the SRS are as follows:
\begin{enumerate}
	\item Online capabilities of the system are not included within the requirements however are alluded to as possible features.
\end{enumerate}
% End SubSection

% End Section


\section{Functional Requirements}
\label{sec:functional_requirements}
% Begin Section
% Begin business event list
\begin{enumerate}[{BE}1.]
   
    %
    % Business Event Template
    %
    % \item BE
    % \begin{enumerate}[{VP1}.1]
    %   \item VP
    %   \begin{enumerate}
    %       \item description
    %   \end{enumerate}
    % \end{enumerate}
    %
   
    %---------------------------%
    %   Business Event
    %---------------------------%  
    \item User wants to start the game
    \begin{enumerate}[{VP1}.1]
      \item User
      \begin{enumerate}
        \item The user shall be able load a previous saved game.
        \item The user shall be able to start a new game by selecting a game mode.
        \item There shall be survival mode and overlord mode avaliable to choose
        \item Upon starting the game, the user shall be periodically prompted with yes-no questions.
        \item The user shall be able to see a loading screen while loading.
      \end{enumerate}
    \end{enumerate}
 
    %---------------------------%
    %   Business Event
    %---------------------------%
    \item User wants to quit the game
    \begin{enumerate}[{VP1}.1]
      \item User
      \begin{enumerate}
        \item The user shall be capable of quitting the game at any time.
        \item The user shall have the option of saving a game at any time.
      \end{enumerate}
    \end{enumerate}
 
    %---------------------------%
    %   Business Event
    %---------------------------%
    \item User wants to respond to yes-no question
    \begin{enumerate}[{VP1}.1]
      \item User
      \begin{enumerate}
        \item The user shall be able to respond to the game by selecting yes or no.
        \item The user shall be presented with a game over screen if losing conditions are fulfilled.
        \item The user shall be presented with a game win screen if winning conditions are fulfilled.
        \item UI elements shall be updated based on the progress made by the user.
      \end{enumerate}
    \end{enumerate}
 
    %---------------------------%
    %   Business Event
    %---------------------------%
    \item User wants to interacts with game view
    \begin{enumerate}[{VP1}.1]
      \item User
      \begin{enumerate}
        \item The user shall be able to move the game view around.
      \end{enumerate}
    \end{enumerate}
 
    %---------------------------%
    %   Business Event
    %---------------------------%
    \item User wants to pause the game
    \begin{enumerate}[{VP1}.1]
      \item User
      \begin{enumerate}
        \item The user shall be able to pause the game at any time he want while the game is running.
      \end{enumerate}
    \end{enumerate}
 
    %---------------------------%
    %   Business Event
    %---------------------------%
    \item User wants to resume the game
    \begin{enumerate}[{VP1}.1]
      \item User
      \begin{enumerate}
        \item The user shall be able to resume the game when the game is paused.
      \end{enumerate}
    \end{enumerate}
 
 
% End busines event list %
\end{enumerate}


% End Section

\section{Non-Functional Requirements}
\label{sec:non-functional_requirements}
% Begin Section

% manually managed enumi variable to preserve enumerations across sections
\newcounter{_enumi}
\newcommand{\holdEnum}{\setcounter{_enumi}{\value{enumi}}}
\newcommand{\resumeEnum}{\setcounter{enumi}{\value{_enumi}}}

\subsection{Look and Feel Requirements}
\label{sub:look_and_feel_requirements}
% Begin SubSection

\subsubsection{Appearance Requirements}
\label{ssub:appearance_requirements}
% Begin SubSubSection
\begin{enumerate}[{LF}1. ]
	\item The product shall minimize the use of visual adornments that would otherwise clutter the interface.
	\item The product shall predominantly use a single colour to help develop and maintain it's brand identity.
	\holdEnum
\end{enumerate}
% End SubSubSection

\subsubsection{Style Requirements}
\label{ssub:style_requirements}
% Begin SubSubSection
\begin{enumerate}[{LF}1.]
	\resumeEnum
	%TODO: Reference the google material design guidelines in section 1.4
	\item The product's application icon shall be designed in the common negative-space style to help it sit better amongst other popular application icons.  
\end{enumerate}
% End SubSubSection

% End SubSection

\subsection{Usability and Humanity Requirements}
\label{sub:usability_and_humanity_requirements}
% Begin SubSection

\subsubsection{Ease of Use Requirements}
\label{ssub:ease_of_use_requirements}
% Begin SubSubSection
\begin{enumerate}[{UH}1. ]
	\item The product shall be usable by anyone fluent in the English language.
	\item The product's navigational elements shall behave in the ways standard to the Android operating system. This should make the application more intuitive to use for new users. 
	\item The product's functions shall be designed a way such that they require a minimum number of gestures from the user.
	\item Where it makes sense, the product shall favour input elements such as sliders over raw text input.
	\holdEnum
\end{enumerate}
% End SubSubSection

\subsubsection{Personalization and Internationalization Requirements}
\label{ssub:personalization_and_internationalization_requirements}
n.a
% Begin SubSubSection
%\begin{enumerate}[{SR}1. ]
%	\resumeEnum
%	\item 
%	\holdEnum
%\end{enumerate}
% End SubSubSection

\subsubsection{Learning Requirements}
\label{ssub:learning_requirements}
% Begin SubSubSection
\begin{enumerate}[{UH}1. ]
	\resumeEnum
	\item By using standard Android navigational elements, the product should not take more than 10 minutes for a new user to become adept with it.
	\holdEnum
\end{enumerate}
% End SubSubSection

\subsubsection{Understandability and Politeness Requirements}
\label{ssub:understandability_and_politeness_requirements}

% Begin SubSubSection
\begin{enumerate}[{UH}1. ]
 	\resumeEnum
	\item The product shall use common navigational elements (ie. status bar, tab bar, hamburger menu) to help usability amongst new users.
	\item The product shall use common iconography to help usability amongst new users. 
	\item The product shall contain no language or ideas inappropriate for an elementary school level user.
	\holdEnum
\end{enumerate}
% End SubSubSection

\subsubsection{Accessibility Requirements}
\label{ssub:accessibility_requirements}
% Begin SubSubSe ction
\begin{enumerate}[{UH}1. ]
	\resumeEnum
	\item The product shall make heavy use of contrast to help it's usability amongst colour blind users.
	\item The product's interface shall use intractable elements at a minimum of 1.5x1.5 cm in size to help users with reduced motor skills.
	\item By employing standard Android SDK best practices as mentioned in LF7, the product shall adapt the user's global accessibility settings such as TTS (text to speech) and increased/decreased font sizes. 
	\item The product shall make heavy use of contrast and minimize the use of multiple colours to help colour blind users (This is also mentioned in LF5).
\end{enumerate}
% End SubSubSection

% End SubSection

\subsection{Performance Requirements}
\label{sub:performance_requirements}
% Begin SubSection

\subsubsection{Speed and Latency Requirements}
\label{ssub:speed_and_latency_requirements}
% Begin SubSubSection
\begin{enumerate}[{PR}1. ]
	\item Each user interaction shall be followed up by a visual change on screen. The time between these two events should never exceed half a second.
	\item Procedural world generation for a new game should take no longer than 2 seconds.
	\item Loading an in-progress game should take no longer than 1 second.
	\item The application should launch in no longer than 2 seconds on a Samsung Galaxy S6.
	\item The application should resume from background in no longer than 1 second on a Samsung Galaxy S6.
	\holdEnum
\end{enumerate}
% End SubSubSection

\subsubsection{Safety-Critical Requirements}
\label{ssub:safety_critical_requirements}
% pretty sure this is n.a - James
n.a
% Begin SubSubSection
%\begin{enumerate}[{PR}1. ]
%	\item 
%\end{enumerate}
% End SubSubSection

\subsubsection{Precision or Accuracy Requirements}
\label{ssub:precision_or_accuracy_requirements}
% Begin SubSubSection
\begin{enumerate}[{PR}1. ]
	\resumeEnum
	\item When performing floating-point arithmetic, the application shall favour the use double precision to minimize rounding error.
	\holdEnum
\end{enumerate}
% End SubSubSection

\subsubsection{Reliability and Availability Requirements}
\label{ssub:reliability_and_availability_requirements}
% Begin SubSubSection
\begin{enumerate}[{PR}1. ]
	\resumeEnum
	\item The product shall not require an active internet connection (ie. it will be available offline).
	\item In any case that the device is running under normal circumstances, the application shall be available.
	\holdEnum
\end{enumerate}
% End SubSubSection

\subsubsection{Robustness or Fault-Tolerance Requirements}
\label{ssub:robustness_or_fault_tolerance_requirements}
% Begin SubSubSection
\begin{enumerate}[{PR}1. ]
	\resumeEnum
	\item The application shall temporarily save an ongoing game in the event that it is terminated unexpectedly by the user or by the operating system in the event that memory needs to be freed. 
	\item The application shall reopen the saved game from PR11 in the event that it is launched following an unexpected termination.
	\holdEnum
\end{enumerate}
% End SubSubSection

\subsubsection{Capacity Requirements}
\label{ssub:capacity_requirements}
n.a
% Begin SubSubSection
%\begin{enumerate}[{SR}1. ]
%	\resumeEnum
%	\item 
%	\holdEnum
%\end{enumerate}
% End SubSubSection

\subsubsection{Scalability or Extensibility Requirements}
\label{ssub:scalability_or_extensibility_requirements}
n.a
% Begin SubSubSection
%\begin{enumerate}[{SR}1. ]
%	\resumeEnum
%	\item 
%	\holdEnum
%\end{enumerate}
% End SubSubSection

\subsubsection{Longevity Requirements}
\label{ssub:longevity_requirements}
% Begin SubSubSection
\begin{enumerate}[{PR}1. ]
	\resumeEnum
	\item The product should be maintainable and extensible within it's planned maintenance budget for 5 years post launch.
\end{enumerate}
% End SubSubSection

% End SubSection

\subsection{Operational and Environmental Requirements}
\label{sub:operational_and_environmental_requirements}
% Begin SubSection

\subsubsection{Expected Physical Environment}
\label{ssub:expected_physical_environment}
% Begin SubSubSection
\begin{enumerate}[{OE}1. ]
	\item As the application operates within the Android environment and introduces no restriction under which it operates within that environment, the product shall be usable under circumstances where the Android operating system allows it.
	\holdEnum
\end{enumerate}
% End SubSubSection

\subsubsection{Requirements for Interfacing with Adjacent Systems}
\label{ssub:requirements_for_interfacing_with_adjacent_systems}
% Begin SubSubSection
\begin{enumerate}[{OE}1. ]
	\resumeEnum
	\item The product shall work on any smartphone running Android 4.4 (KitKat) and later.
	\holdEnum
\end{enumerate}
% End SubSubSection

\subsubsection{Productization Requirements}
\label{ssub:productization_requirements}
% Begin SubSubSection
\begin{enumerate}[{OE}1. ]
	\resumeEnum
	% TODO: Link to the google play store in section 1.4.
	\item The product shall be installable directly from the Google Play Store.
	\item The product shall have 5 screenshots and a short text description to advertise it on the Google Play Store.
	\holdEnum
\end{enumerate}
% End SubSubSection

\subsubsection{Release Requirements}
\label{ssub:release_requirements}
% Begin SubSubSection
\begin{enumerate}[{OE}1. ]
	\resumeEnum
	\item Maintenance and new content updates will be published at the end of each month.
	\holdEnum
\end{enumerate}
% End SubSubSection

% End SubSection

\subsection{Maintainability and Support Requirements}
\label{sub:maintainability_and_support_requirements}
% Begin SubSection

\subsubsection{Maintenance Requirements}
\label{ssub:maintenance_requirements}
% Begin SubSubSection
\begin{enumerate}[{MS}1. ]
	\item Hotfixes must available to publish within a day of them being made.
	\item The code shall be architected modularly such that it can be more easily maintained.
	\holdEnum
\end{enumerate}
% End SubSubSection

\subsubsection{Supportability Requirements}
\label{ssub:supportability_requirements}
% Begin SubSubSection
\begin{enumerate}[{MS}1. ]
	\resumeEnum
	% TODO: Link to the google play store in 1.4.
	\item A link to the company's support url will be provided in the description on the Google Play Store for feature requests and bug reports.
	\holdEnum
\end{enumerate}
% End SubSubSection

\subsubsection{Adaptability Requirements}
\label{ssub:adaptability_requirements}
% Begin SubSubSection
\begin{enumerate}[{MS}1. ]
	\resumeEnum
	\item The product's interface shall be designed in a way such that it can be easily adapted to different screen sizes.
	\item The product is expected to run on version of Android 4.4 (KitKat) and later.
\end{enumerate}
% End SubSubSection

% End SubSection

\subsection{Security Requirements}
\label{sub:security_requirements}
% Begin SubSection

\subsubsection{Access Requirements}
\label{ssub:access_requirements}
% Begin SubSubSection
\begin{enumerate}[{SR}1. ]
	\item When online features are introduced in the future, only a user logged into an account may play under that account's identity.
	\holdEnum
\end{enumerate}
% End SubSubSection

\subsubsection{Integrity Requirements}
\label{ssub:integrity_requirements}
% Begin SubSubSection
\begin{enumerate}[{SR}1. ]
	\resumeEnum
	\item The product shall guard against the introduction of incorrect data to, for example, guard against an attacker spoofing high scores numbers.
	\item When online features are introduced in the future, the product shall guard against the automated generation of user accounts. 
	\holdEnum
\end{enumerate}
% End SubSubSection

\subsubsection{Privacy Requirements}
\label{ssub:privacy_requirements}
% Begin SubSubSection
\begin{enumerate}[{SR}1. ]
	\resumeEnum
	\item The product shall inform users of its information policy before collecting data on them.
	\item The product shall inform the users of changes to its information policy.
	\item The product shall adhere to it's information policy if revealing private user information.
	\holdEnum
\end{enumerate}
% End SubSubSection

\subsubsection{Audit Requirements}
\label{ssub:audit_requirements}
n.a
% Begin SubSubSection
%\begin{enumerate}[{SR}1. ]
%	\resumeEnum
%	\item 
%	\holdEnum
%\end{enumerate}
% End SubSubSection

\subsubsection{Immunity Requirements}
\label{ssub:immunity_requirements}

% Begin SubSubSection
\begin{enumerate}[{SR}1. ]
	\resumeEnum
	\item As the product runs under management of the Android operating system, the immunity of the product relies on the guards put in place by Android.
\end{enumerate}
% End SubSubSection

% End SubSection

\subsection{Cultural and Political Requirements}
\label{sub:cultural_and_political_requirements}
% Begin SubSection

\subsubsection{Cultural Requirements}
\label{ssub:cultural_requirements}
% Begin SubSubSection
\begin{enumerate}[{CP}1. ]
	\item The product shall favour the use of icons such that, when needed, it can be more easily adapted to different languages and cultures.
	\item The product shall aim to not offend any ethnic or religious groups.
	\holdEnum
\end{enumerate}
% End SubSubSection

\subsubsection{Political Requirements}
\label{ssub:political_requirements}
n.a
% Begin SubSubSection
%\begin{enumerate}[{CP}1. ]
%	\item 
%\end{enumerate}
% End SubSubSection

% End SubSection

\subsection{Legal Requirements}
\label{sub:legal_requirements}
% Begin SubSection

\subsubsection{Compliance Requirements}
\label{ssub:compliance_requirements}
% Begin SubSubSection
\begin{enumerate}[{LR}1. ]
	\item Personal information shall be implemented in compliance with the data protection act.
	\holdEnum
\end{enumerate}
% End SubSubSection

\subsubsection{Standards Requirements}
\label{ssub:standards_requirements}
% Begin SubSubSection
\begin{enumerate}[{LR}1. ]
	\resumeEnum
	\item The product shall use Android SDK best practices so that for instance the user's global accessibility settings (text size, text-to-speech etc...) will take effect in the app. 
	\item Where it can, the product shall conform to Google's material design guidelines.
	\item The application will conform with acceptable performance standards for games and simulations.
	\item The application will conform to the Android Core Application Quality standards to minimize turnover approval time and to increase the chances of being featured in the Google Play Store.
\end{enumerate}
% End SubSubSection

% End SubSection

% End Section
\newpage
\appendix
\section{Division of Labour}
\label{sec:division_of_labour}
% Begin Section
Division of Labour sheet which indicating the contributions of each team member. This sheet must be signed by all team members.
\\
\\
\begin{tabular}{ | l | l | }
\hline
	\textbf{Contributor Name} & \textbf{Contributions}  \\
  	\hline
  	Wenbin Yuan & Section 3.* Functional Requirements Planning and Formalization in association with Haris\\ 		\hline
  	Haris Khan & Section 3.* Functional Requirements Planning and Formalization in association with Wenbin \\
  	\hline
  	Riley McGee & Section 2.* Overall Description
  	\\
  	\hline
  	Vishesh Gulatee & Section 1.* Introduction
  	\\
  	\hline
  	James Taylor & Section 4.* Non-Functional Requirements
  	\\
  	\hline
\end{tabular}
\\
\\
By signing below you agree to the work divisions stated above correctly representing all contributions made:

% End Section
\nocite{MDes}
\nocite{CAQ}
\newpage
\section*{IMPORTANT NOTES}
\begin{itemize}
	\item Be sure to include all sections of the template in your document regardless whether you have something to write for each or not
	\begin{itemize}
		\item If you do not have anything to write in a section, indicate this by the \emph{N/A}, \emph{void}, \emph{none}, etc.
	\end{itemize}
	\item Uniquely number each of your requirements for easy identification and cross-referencing
	\item Highlight terms that are defined in Section~1.3 (\textbf{Definitions, Acronyms, and Abbreviations}) with \textbf{bold}, \emph{italic} or \underline{underline}
	\item For Deliverable 1, please highlight, in some fashion, all (you may have more than one) creative and innovative features. Your creative and innovative features will generally be described in Section~2.2 (\textbf{Product Functions}), but it will depend on the type of creative or innovative features you are including.
\end{itemize}


\end{document}
%------------------------------------------------------------------------------